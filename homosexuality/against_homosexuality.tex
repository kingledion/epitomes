\documentclass[10pt]{article}
\usepackage[margin=1.0in]{geometry}
\usepackage{graphicx}
\usepackage{setspace}

%\usepackage{textgreek}
%\usepackage[LGR,T1]{fontenc}
%\usepackage[utf8]{inputenc}   % utf8 is required

%\newcommand{\textgreek}[1]{\begingroup\fontencoding{LGR}\selectfont#1\endgroup}
%\usepackage[greek]{babel}

\setlength{\parskip}{\baselineskip}%
\setlength{\parindent}{0pt}%
%\renewcommand{\baselinestretch}{1.4}

\let\oldquote\quote
\let\endoldquote\endquote
\renewenvironment{quote}[2][]
  {\if\relax\detokenize{#1}\relax
     \def\quoteauthor{#2}%
   \else
     \def\quoteauthor{#2~---~#1}%
   \fi
   \oldquote}
  {\par\nobreak\smallskip\hfill\textit{\quoteauthor}%
   \endoldquote\addvspace{\bigskipamount}}

\begin{document}
\begin{titlepage}\centering
{\scshape\huge Biblical Guidance on Homosexuality \par}
\vfill
{\scshape\large Daniel Hartig \par}
\vfill
\end{titlepage}
\singlespace
\section*{Introduction}

The Bible's assertions regarding homosexuality are clearly stated, without contradiction, through both the Old and New Testaments. This text is written specifically as a rebutal to \textit{The Bible, Christianity, and Homosexuality} by Justin R. Cannon, and fills the gaps in that work where the author obfusticates the truth by omission. It would be wise for us to remember,
\begin{quote}{2 Peter 2:1-2}
But there were also false prophets among the people evan as there will be false teachers among you, who will secretly bring in destructive heresies, even denying the Lord who bought them, and bring on themselves swift destruction. And many will follow their destructive ways, because of whom the way of truth will be blasphemed. 
\end{quote} 
But against false teachers, 
\begin{quote}{2 Timothy 3:16}
All scrpture is given by inspiration of God, and is profitable for doctrine, for reproof, for correction, for instruction in righetousness.
\end{quote} 
Let us consider Scripture with humility.

\subsubsection*{A note on Translations}

For clarity, I will compare the three most commonly used English Bible translations, not counting the King James Version (KJV), which has acknowledged deficiencies. The New Revised Standard Version of 1989 is a descendent of the KJV (1611), by way of the Revised Version (1885), American Standard Version (1901), and Revised Standard Version (1952). The other two translations are the New International Version (NIV), an independant translation of 1978 updated in 1984; and the New American Bible, Revised Edition (NABRE), a Catholic translation of 2011. The NABRE derives from the New American Bible (1970), and the Douay-Rheims Bible (1610). 

The Greek text of the New Testament comes from \textit{The Greek New Testament According to the Majority Text, Second Edition}, Ed. Zane C. Hodges and Arthur L. Farstad, 1985. The \textit{Septuagint} is found from https://www.ellopos.net/elpenor/greek-texts/septuagint/default.asp; along with an English translation by L.C.L. Brenton. Greek translations are done by the author using \textit{The New Analytical Greek Lexicon}, Wesley J. Perschbacher, 1990 and \textit{New Testament Greek: A Beginning and Intermediate Grammar}, James Allen Hewett, 1986.


\section*{Terminology}

\subsubsection*{Homosexual}

The word \textit{homosexual} does not appear in NRSV or NABRE. In the NIV it appears in 1 Timothy 1:10 as a translation of \textit{ἀρσενοκοίταις} (arsenokoitais), as discussed separately, below. Therefore, widely-read translations of the Bible do not use the word \textit{homosexual}, except to translate one specific word.

\subsubsection*{Sodomite}

\indent The world \textit{sodomite} is used in two places in NRSV (1 Cor 6:9, 1 Tim 1:10), both times as a translation for \textit{ἀρσενοκοίταις} (arsenokoitais), as discussed separately, below. The NABRE uses \textit{sodomite} in the same two cases. The NIV does not use the word at all

None of the above translations uses the word \textit{sodomy}. Therefore, widely-read translations of the Bible do not use the words \textit{sodomy} or \textit{sodomy}, except to translate one specific word.

\subsubsection*{Arsenokoites (ἀρσενοκοίτης)}

This word is used twice in the Bible, both times by Paul in 1 Corinthians 6:9 and 1 Timothy 1:10. The word is not attested in use before Paul's writing (perhaps ~55 AD for 1 Corinthians), so the word's meaning is somewhat uncertain. 

The best evidence for Paul's usage of the word is its likely source, the Septuagint text of Leviticus 18:22.

\begin{quote}{Septuagint}
καὶ μετὰ ἄρσενος οὐ κοιμηθήσῃ κοίτην γυναικείαν, βδέλυγμα γάρ ἐστι. 
\end{quote}
\begin{quote}{Transliteration}
Kai meta arsenos ou koimethese koiten gynaikeian, bdelugma gar esti.
\end{quote}
\begin{quote}{Literal translation}
And with men[pl.] not (you will have fallen asleep)[2nd. pass. sing. fut. ind.] (in bed) [acc.] womanly[acc.], abomination for (it is)[3rd sing. pres. ind.]
\end{quote}
For detailed notes on the author's literal translation, please see Appendix \ref{A}. The common translations of this verse are
\begin{quote}{NRSV}
You shall not lie with a male as with a woman; it is an abomination.
\end{quote}
\begin{quote}{NABRE}
You shall not lie with a male as with a woman; such a thing is an abomination.
\end{quote}
\begin{quote}{NIV}
Do not have sexual relations with a man as one does with a woman; that is detestable.
\end{quote}
\pagebreak
The word \textit{arsenokoites} can be easily explained as Paul using the words \textit{arsenos}  and \textit{koiten} to contract the Greek phrase \textit{meta arsenos koimethese koiten gynaikeian}. The two roots taken together and pluralized, yield \textit{arsenokoites}. This is comparable to Shakespeare's invention of compound words such as `eyeball,' `dishearten,' and `eventful.'

Paul's use of this term is thus a direct link to the text of Leviticus, and the best interpretation of the words meaning is the noun that is an object of that passage. Since that passage is translated as forbidding men from lying with other men as they would a woman, it is most reasonable to assume that Paul uses the term \textit{arsenokoites} to refer to such men. 

\section*{Passage I: The Sodom Account}


\appendix
\subsection{Derivation of the meaning of Leviticus 18:22 in the Sepuagint}\label{A}
The base of the word transliterated as \textit{koimethese} is \textit{koimao}, which means literaly, to fall asleep, as when the Chief Priests order the tomb guards to report that disciples stole Jesus' body while they were asleep (Matt 28:13); and when the disciples slept at the Mount of Olives before Jesus' arrest (Luke 22:45). It is alternately used as a euphaism for death, as in the death of Stephen the martyr (Acts 7:60) and in a speech of Paul's when he describes that David `fell asleep' and `was laid beside his ancestors' (NRSV).

In this usage in Levitacus, the verb has the second person, is passive and singular, has the indicative voice, and future tense; so `you will have fallen asleep'. The word transliterated \textit{koiten} is the accusative case, meaning it is the recipient of the action of the verb; the falling asleep happens in bed. The adjective \textit{gynaikeian} is accusative, meaning it is operating on the object of the verb, that is, the bed; it is the bed that is womanly. The final verb \textit{esti} is `to be'; in the third person, singuar, with indicative voice and present tense this is `it is'.

\end{document}