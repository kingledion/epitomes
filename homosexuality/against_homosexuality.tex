\documentclass[10pt]{article}
\usepackage[margin=1.0in]{geometry}
\usepackage{graphicx}
\usepackage{setspace}

%\usepackage{textgreek}
%\usepackage[LGR,T1]{fontenc}
%\usepackage[utf8]{inputenc}   % utf8 is required

%\newcommand{\textgreek}[1]{\begingroup\fontencoding{LGR}\selectfont#1\endgroup}
%\usepackage[greek]{babel}

\setlength{\parskip}{\baselineskip}%
\setlength{\parindent}{0pt}%
%\renewcommand{\baselinestretch}{1.4}

\let\oldquote\quote
\let\endoldquote\endquote
\renewenvironment{quote}[2][]
  {\if\relax\detokenize{#1}\relax
     \def\quoteauthor{#2}%
   \else
     \def\quoteauthor{#2~---~#1}%
   \fi
   \oldquote}
  {\par\nobreak\smallskip\hfill\textit{\quoteauthor}%
   \endoldquote\addvspace{\bigskipamount}}

\begin{document}
\begin{titlepage}\centering
\vfill
{\scshape\huge Biblical Guidance on Homosexuality \par}
\vfill
{\scshape\large Daniel Hartig \par}
\vfill
\end{titlepage}
\singlespace
\section*{Introduction}

The Bible's assertions regarding homosexuality are clearly stated, without contradiction, through both the Old and New Testaments. This text is written specifically as a rebutal to \textit{The Bible, Christianity, and Homosexuality} by Justin R. Cannon, and fills the gaps in that work where the author obfusticates the truth by omission. It would be wise for us to remember,
\begin{quote}{2 Peter 2:1-2}
But there were also false prophets among the people evan as there will be false teachers among you, who will secretly bring in destructive heresies, even denying the Lord who bought them, and bring on themselves swift destruction. And many will follow their destructive ways, because of whom the way of truth will be blasphemed. 
\end{quote} 
But against false teachers, 
\begin{quote}{2 Timothy 3:16}
All scrpture is given by inspiration of God, and is profitable for doctrine, for reproof, for correction, for instruction in righetousness.
\end{quote} 
Let us consider Scripture with humility.

\subsubsection*{A note on Translations}

For clarity, I will compare the three most commonly used English Bible translations, not counting the King James Version (KJV), which has acknowledged deficiencies. The New Revised Standard Version of 1989 is a descendent of the KJV (1611), by way of the Revised Version (1885), American Standard Version (1901), and Revised Standard Version (1952). The other two translations are the New International Version (NIV), an independant translation of 1978 updated in 1984; and the New American Bible, Revised Edition (NABRE), a Catholic translation of 2011. The NABRE derives from the New American Bible (1970), and the Douay-Rheims Bible (1610). 

The Greek text of the New Testament comes from \textit{The Greek New Testament According to the Majority Text, Second Edition}, Ed. Zane C. Hodges and Arthur L. Farstad, 1985. The \textit{Septuagint} is found from https://www.ellopos.net/elpenor/greek-texts/septuagint/default.asp; along with an English translation by L.C.L. Brenton. Greek translations are done by the author using \textit{The New Analytical Greek Lexicon}, Wesley J. Perschbacher, 1990 and \textit{New Testament Greek: A Beginning and Intermediate Grammar}, James Allen Hewett, 1986.


\section*{Terminology}

\subsubsection*{Homosexual}

The word \textit{homosexual} does not appear in NRSV or NABRE. In the NIV it appears in 1 Timothy 1:10 as a translation of \textit{ἀρσενοκοίταις} (arsenokoitais), as discussed separately, below. Therefore, widely-read translations of the Bible do not use the word \textit{homosexual}, except to translate one specific word.

\subsubsection*{Sodomite}

\indent The world \textit{sodomite} is used in two places in NRSV (1 Cor 6:9, 1 Tim 1:10), both times as a translation for \textit{ἀρσενοκοίταις} (arsenokoitais), as discussed separately, below. The NABRE uses \textit{sodomite} in the same two cases. The NIV does not use the word at all

None of the above translations uses the word \textit{sodomy}. Therefore, widely-read translations of the Bible do not use the words \textit{sodomy} or \textit{sodomy}, except to translate one specific word.

\subsubsection*{Arsenokoites (ἀρσενοκοίτης)}

This word is used twice in the Bible, both times by Paul in 1 Corinthians 6:9 and 1 Timothy 1:10. The word is not attested in use before Paul's writing (perhaps ~55 AD for 1 Corinthians), so the word's meaning is somewhat uncertain. 

The best evidence for Paul's usage of the word is its likely source, the Septuagint text of Leviticus 18:22.

\begin{quote}{Septuagint text in Koine Greek}
καὶ μετὰ ἄρσενος οὐ κοιμηθήσῃ κοίτην γυναικείαν, βδέλυγμα γάρ ἐστι. 
\end{quote}
\begin{quote}{Transliteration}
Kai meta arsenos ou koimethese koiten gynaikeian, bdelugma gar esti.
\end{quote}
\begin{quote}{Literal translation}
And with men[pl.] not (you will have fallen asleep)[2nd. pass. sing. fut. ind.] (in bed) [acc.] womanly[acc.], abomination for (it is)[3rd sing. pres. ind.]
\end{quote}
For detailed notes on the author's literal translation, please see Appendix \ref{A}. The common translations of this verse are
\begin{quote}{Translations}
You shall not lie with a male as with a woman; it is an abomination. (NRSV)\linebreak
You shall not lie with a male as with a woman; such a thing is an abomination. (NABRE)\linebreak
Do not have sexual relations with a man as one does with a woman; that is detestable. (NIV)\linebreak
\end{quote}
The word \textit{arsenokoites} can be explained as Paul using the words \textit{arsenos}  and \textit{koiten} to contract the Greek phrase \textit{meta arsenos koimethese koiten gynaikeian}. The two roots taken together and pluralized, yield \textit{arsenokoites}. This is comparable to Shakespeare's invention of compound words such as `eyeball,' `dishearten,' and `eventful.'

Paul's use of this term is thus a direct link to the text of Leviticus, and the Levitical text is the best interpretation of the word's meaning. Since that passage is translated as forbidding men from lying with other men as they would a woman, it is most reasonable to translate the term \textit{arsenokoites} as referring to such men. 

\section*{Passage I: The Sodom Account}

There are many references to Sodom in the Bible and the deuterocanonical texts, quoted here in the NRSV translation. First are the `historical' references to the city before its destruction. The first reference, Genesis 14:1-14 discusses the war in which Sodom (and Gomorrah) participated participated. Sodom was on the losing side of the war; Lot, nephew of Abraham, was carried off as spoils of war when Sodom was sacked by its victorious enemies. Abraham lead his men to the rescue of Lot. No particular sinfulness is ascribed to Sodom at this time. 

The next reference is to the depravity of Sodom, and its eventual destruction. First, the Lord declares that Sodom is wicked and determines that He will investigate.
\begin{quote}{Genesis 18:20-21}
Then the Lord said,``How great is the outcry against Sodom and Gomorrah and how very grave their sin! I must go down and see whether they have done altogether according to the outcry that has come to me; and if not, I will know''
\end{quote}
Next, Abraham pleads with the Lord to spare the city if righteous men can be found, bargaining with the Lord to spare the city if ten righteous men can be found in it.
\begin{quote}{Genesis 18:32}
Then he said, ``Oh do not let the Lord be angry if I speak just once more. Suppose ten are found there.” He answered, “For the sake of ten I will not destroy it.''
\end{quote}
Finally, the angels that God has sent into the city are confronted by an angry crowd
\begin{quote}{Genesis 19:45}
But before they lay down, the men of the city, the men of Sodom, both young and old, all the people to the last man, surrounded the house; and they called to Lot, ``Where are the men who came to you tonight? Bring them out to us, so that we may know them.''
\end{quote}
The `historical' account of the destruction of Sodom ascribes to them many sins, but when the Lord sends his messengers to Sodom to investigate, they are specifically confronted with a threat of non-consensual, homosexual sex. 

After the details of what happened to spur the city's destruction, many later biblical and extra biblical authors discussed the crimes of the city, offering many explanations for what the inhabitants did wrong.
\begin{quote}{Isaiah 3:9}
The look on their faces bears witness against them;
    they proclaim their sin like Sodom,
    they do not hide it.
Woe to them!
    For they have brought evil on themselves.
\end{quote}
\begin{quote}{Jeremiah 23:14}
But in the prophets of Jerusalem
    I have seen a more shocking thing:
they commit adultery and walk in lies;
    they strengthen the hands of evildoers,
    so that no one turns from wickedness;
all of them have become like Sodom to me,
    and its inhabitants like Gomorrah.
\end{quote}
\begin{quote}{Ezekiel 16:49-50}
Now this was the sin of your sister Sodom: She and her daughters were arrogant, overfed and unconcerned; they did not help the poor and needy. They were haughty and did detestable things before me. Therefore I did away with them as you have seen.
\end{quote}
\begin{quote}{Jude 1:7}
Likewise, Sodom and Gomorrah and the surrounding cities, which, in the same manner as they, indulged in sexual immorality and pursued unnatural lust, serve as an example by undergoing a punishment of eternal fire.
\end{quote}
The phrase `pursuing unnatural lust` deserves further attention. The original Greek text is \textit{ἀπελθοῦσαι ὀπίσω σαρκὸς ἑτέρας}; transliterated as \textit{apelthousai opiso sarkos eteras}. The base of word \textit{apelthousai} means `to go forth', the verb is used as an nominative, aorist, active participle. Aorist case means that the word refers to past events that still affect the present. A particple is a verb used as and adjective, while the nominative case refers to a word that is related to the subject of a sentence.  The next word, \textit{opiso} is an adverb meaning `behind' or `after'. The word for the human body is \textit{sarkh}, where the `kh` is the Greek letter Xi; \textit{sarkos} is the genitive case of this word. The genitive case refers to a noun or adjective acts as a description of another word. In this case it is describing the verb `to go forth'. Finally \textit{eteras} is the genitive form of the adjective `other'. 

In the sentence of Jude 1:7, the simple subject-verb combination is Sodom and Gomorrah serve. Thus, \textit{apelthousai} describes Sodom as `having gone forth in the past.' The remaining words form a prepositional phrase describing the goings forth, they (the Sodomites) were going forth after other bodies. In the context of the story from Genesis, the other bodies is that of the messengers who were with Lot. Translations of this phrase include `pursued unnatural lust' in NRSV, `practiced unnatural vice' in NABRE, and `perversion' in NIV. 

In addition to specific mentions of sin, there are a great many passages in the Bible comparing a threatened punishment with the fate of Sodom and Gomorrah. None of these passages associate Sodom with any particular sin, but instead warn that the fate of the sinful will be like (or worse than) Sodom and Gomorrah. These passages include Deuteronomy 29:23, Isaiah 1:9-10, Isaiah 13:19, Jeremiah 49:18, Jeremiah 50:40, Lamentations 4:6, Amos 4:11, Zephaniah 2:9, Matthew 10:15, Matthew 11:24, Luke 10:12, Luke 17:29, Romans 9:29 (which is a quotation by Paul of the aforementioned Isaiah 1:9), 2 Peter 2:6. The deuterocanonical books add to this Wisdom 10:6, Wisdom 19:17, and Sirach 16:8. Lastly, there is an allegorical reference to Sodom in Revelation 11:8. 

The ennumerated sins of Sodom found in the Bible thus include, pride in sinning, adultery, lying, assisting evildoers, arogance, gluttony, apathy, failure to assist the poor, sexual immorality, and `pursuing unnatural lust', in additional to the vignette of rape and male homosexual sex described in Genesis. Surely, the Lord is correct when he says of Sodom, `how very grave is their sin'!

Of the many sins of Sodom, the (attempted) homosexual intercourse described in the story of its destruction was the most impactful; no where else in the Bible is such a story found. The rape aspect of the incident, both threatened on the messengers and committed on Lot's daughters is not unique in the Bible, as there are stories of rapes of Tamar and a Levite concubine. Indeed, it can be argued that Lot's daughters, subjected to rape in the tale of Sodom, in turn rape their own father later. The encounter at Sodom lead directly to the destruction of the city and that the destruction of the city was mentioned 14 (or 17) additional times in the scripture. This explains why homosexual intercourse is strongly associated with the story of Sodom, even though sexual perversion was only one of many sins of that city.

Mention Philo and Josephus. 

\appendix
\subsection{Derivation of the meaning of Leviticus 18:22 in the Sepuagint}\label{A}
The base of the word transliterated as \textit{koimethese} is \textit{koimao}, which means literaly, to fall asleep, as when the Chief Priests order the tomb guards to report that disciples stole Jesus' body while they were asleep (Matt 28:13); and when the disciples slept at the Mount of Olives before Jesus' arrest (Luke 22:45). It is alternately used as a euphaism for death, as in the death of Stephen the martyr (Acts 7:60) and in a speech of Paul's when he describes that David `fell asleep' and `was laid beside his ancestors' (NRSV).

In this usage in Levitacus, the verb has the second person, is passive and singular, has the indicative voice, and future tense; so `you will have fallen asleep'. The word transliterated \textit{koiten} is the accusative case, meaning it is the recipient of the action of the verb; the falling asleep happens in bed. The adjective \textit{gynaikeian} is accusative, meaning it is operating on the object of the verb, that is, the bed; it is the bed that is womanly. The final verb \textit{esti} is `to be'; in the third person, singuar, with indicative voice and present tense this is `it is'.

\end{document}