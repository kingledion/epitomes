\documentclass[10pt]{article}
\usepackage[margin=1.0in]{geometry}
\usepackage{graphicx}
\usepackage{setspace}

%\usepackage[T1]{fontenc}
\usepackage[utf8x]{inputenc}
%\usepackage[greek,english]{babel}

\usepackage{textgreek}
%\usepackage[LGR,T1]{fontenc}
%\usepackage[utf8]{inputenc}   % utf8 is required

%\newcommand{\textgreek}[1]{\begingroup\fontencoding{LGR}\selectfont#1\endgroup}
%\usepackage[greek]{babel}

\setlength{\parskip}{\baselineskip}%
\setlength{\parindent}{0pt}%
%\renewcommand{\baselinestretch}{1.4}

\let\oldquote\quote
\let\endoldquote\endquote
\renewenvironment{quote}[2][]
  {\if\relax\detokenize{#1}\relax
     \def\quoteauthor{#2}%
   \else
     \def\quoteauthor{#2~---~#1}%
   \fi
   \oldquote}
  {\par\nobreak\smallskip\hfill\textit{\quoteauthor}%
   \endoldquote\addvspace{\bigskipamount}}

\begin{document}
%\begin{titlepage}\centering
%\null\vskip 2in
%{\scshape\huge Biblical Guidance on Homosexuality \par}
%\vfill
%{\scshape\large Daniel Hartig \par}
%\vfill
%\end{titlepage}
\singlespace
%\section*{Introduction: A Note on Translations}

%For clarity, I will compare the three most commonly used English Bible translations, not counting the King James Version (KJV), which has acknowledged deficiencies. The New Revised Standard Version of 1989 is a descendent of the KJV (1611), by way of the Revised Version (1885), American Standard Version (1901), and Revised Standard Version (1952). The other two translations are the New International Version (NIV), an independant translation of 1978 updated in 1984; and the New American Bible, Revised Edition (NABRE), a Catholic translation of 2011. The NABRE derives from the New American Bible (1970), and the Douay-Rheims Bible (1610). 

%The Greek text of the New Testament comes from \textit{The Greek New Testament According to the Majority Text, Second Edition}, Ed. Zane C. Hodges and Arthur L. Farstad, 1985. The \textit{Septuagint} is found from https://www.ellopos.net/elpenor/greek-texts/septuagint/default.asp; along with an English translation by L.C.L. Brenton. Greek translations are done by the author using \textit{The New Analytical Greek Lexicon}, Wesley J. Perschbacher, 1990 and \textit{New Testament Greek: A Beginning and Intermediate Grammar}, James Allen Hewett, 1986.


%\section*{Use of Modern Terminology in Translations}

%\subsubsection*{Homosexual}

%The word \textit{homosexual} does not appear in NRSV or NABRE. In the NIV it appears in 1 Timothy 1:10 as a translation of {\textgreekfont ἀρσενοκοίταις} (arsenokoitais), as discussed separately, below. Therefore, widely-read translations of the Bible do not use the word \textit{homosexual}, except to translate one specific word.

%\subsubsection*{Sodomite}

%\indent The world \textit{sodomite} is used in two places in NRSV (1 Cor 6:9, 1 Tim 1:10), both times as a translation for {\textgreekfont ἀρσενοκοίταις} (arsenokoitais), as discussed separately, below. The NABRE uses \textit{sodomite} in the same two cases. The NIV does not use the word at all. None of the above translations uses the word \textit{sodomy}. Therefore, widely-read translations of the Bible do not use the words \textit{sodomy} or \textit{sodomy}, except to translate one specific word.

\section*{Proscriptions in the Law}

The prohibition of homosexual activities between men is found in Levitius 18:22:
\begin{quote}{Leviticus 18:22, NRSV}
You shall not lie with a male as with a woman; it is an abomination. 
\end{quote}
That this command is in force for Jewish law is in little doubt. But for Christians, following the letter of the Mosaic law has not been a requirement, as expounded in the New Testament:
\begin{quote}{Acts 15:19-20, NRSV}
Therefore I have reached the decision that we should not trouble those Gentiles who are turning to God, but we should write to them to abstain only from things polluted by idols and from fornication and from whatever has been strangled[e] and from blood.
\end{quote}
Jesus himself explicity rejects parts of the Mosaic law;
% Mosaic divorce from Deuteronomy 24:1-2 is rejected by Jesus in Mark 10:5-9 and other places; 
retributional justice ("an eye for an eye, a tooth for a tooth") from Exodus 24:23-25 and other places is rejected by Jesus in Matthew 5:38-39. Dietary restrictions and association with non-Jews are rejected by divine revelation to Peter in Acts 10:11-16. 

Yet, while Mosaic law is not to be retained intact by Christians, there is still a Law ordained by God that Christians are to obey. Jesus says that 
\begin{quote}{Matthew 5:17-20}
Do not think that I have come to abolish the Law or the Prophets; I have not. Do not think that I have come to abolish the law or the prophets; I have come not to abolish but to fulfill. For truly I tell you, until heaven and earth pass away, not one letter, not one stroke of a letter, will pass from the law until all is accomplished. Therefore, whoever breaks one of the least of these commandments, and teaches others to do the same, will be called least in the kingdom of heaven; but whoever does them and teaches them will be called great in the kingdom of heaven. For I tell you, unless your righteousness exceeds that of the scribes and Pharisees, you will never enter the kingdom of heaven.
\end{quote}
This is theologically expounded upon by Paul:
%\begin{quote}{Romans 3:21-26}
%But now, apart from law, the righteousness of God has been disclosed, and is attested by the law and the prophets, the righteousness of God through faith in Jesus Christ for all who believe. For there is no distinction, since all have sinned and fall short of the glory of God; they are now justified by his grace as a gift, through the redemption that is in Christ Jesus, whom God put forward as a sacrifice of atonement by his blood, effective through faith. He did this to show his righteousness, because in his divine forbearance he had passed over the sins previously committed; it was to prove at the present time that he himself is righteous and that he justifies the one who has faith in Jesus
%\end{quote}
\begin{quote}{Romans 8:1-6}
There is therefore now no condemnation for those who are in Christ Jesus. For the law of the Spirit of life in Christ Jesus has set you free from the law of sin and of death. For God has done what the law, weakened by the flesh, could not do: by sending his own Son in the likeness of sinful flesh, and to deal with sin, he condemned sin in the flesh, so that the just requirement of the law might be fulfilled in us, who walk not according to the flesh but according to the Spirit. For those who live according to the flesh set their minds on the things of the flesh, but those who live according to the Spirit set their minds on the things of the Spirit. To set the mind on the flesh is death, but to set the mind on the Spirit is life and peace. 
\end{quote}
All sin and disobey the Mosaic law, even the most rigorous of Pharisees. Therefore, all are condemned by the law, as part of our human nature, afflicted by the weakness of the flesh. Yet, if Mosaic law is associated with the flesh,  we can follow, through faith in Christ, the law of the Spirit and find life and peace. What is this law of the spirit? Jesus declares that this law of the Spirit is the Mosaic law fulfilled through him; this law is the disclosd righteousness of God, according to Paul. While the law of the Spirit is not enumerated as the old Mosaic law was, the New Tetament abounds with descriptions of sins to avoid and virtues to pursue. 

Thus, parts of Mosiac law clearly remain in force. There is no doubt that murder, theft, and adultery are still sinful and neighborly love and charity are still virtues. Paul is clear that those who ignore the law of the Spirit and set their mind on the flesh will only find death. Paul only repeats Jesus' teaching that love for the Father is linked to commandments:
\begin{quote}{John 14:21, NRSV}
They who have my commandments and keep them are those who love me; and those who love me will be loved by my Father, and I will love them and reveal myself to them.
\end{quote}
\begin{quote}{John 5:28-29, NRSV}
Do not be astonished at this; for the hour is coming when all who are in their graves will hear his voice and will come out—those who have done good, to the resurrection of life, and those who have done evil, to the resurrection of condemnation.
\end{quote}

Thus, some proscriptions from Levitical law are no longer commandments that Christians must follow; but some still are. We must use the evidence of the New Testament to see if the Levitical proscription of sexual intercourse between men is explicitly incorporated into the law of the Spirit. 


\section*{Arsenokoites ({\textgreekfont ἀρσενοκοίτης)}}

This word is used twice in the Bible, both times by Paul in 1 Corinthians 6:9 and 1 Timothy 1:10. The word is not attested in use before Paul's writing (perhaps ~55 AD for 1 Corinthians), so the word's meaning is somewhat uncertain. 

The best evidence for the meaning of Paul's word is its likely source, the Septuagint text of Leviticus 18:22. The Septuagint is a translation of the Hebrew Bible into Koine Greek, first done about 250 BC.
\begin{quote}{Septuagint text of Leviticus 18:22, with transliteration}
{\textgreekfont \textKappa αὶ μετὰ ἄρσενος οὐ κοιμηθήσῃ κοίτην γυναικείαν, βδέλυγμα γάρ ἐστι.}\\Kai meta arsenos ou koimethese koiten gynaikeian, bdelugma gar esti.
\end{quote}
%\begin{quote}{Literal translation}
%And with men[pl.] not (you will have fallen asleep)[2nd. pass. sing. fut. ind.] (in bed) [acc.] womanly[acc.], abomination for (it is)[3rd sing. pres. ind.]
%\end{quote}
%For detailed notes on the author's literal translation, please see Appendix \ref{A}. 
%The common translations of this verse are
%\begin{quote}{Translations}
%You shall not lie with a male as with a woman; it is an abomination. (NRSV)\linebreak
%You shall not lie with a male as with a woman; such a thing is an abomination. (NABRE)\linebreak
%Do not have sexual relations with a man as one does with a woman; that is detestable. (NIV)\linebreak
%\end{quote}
The word \textit{arsenokoites} can be explained as Paul using the words \textit{arsenos}  and \textit{koiten} to contract the Greek phrase \textit{meta arsenos...koimethese koiten gynaikeian}, which could be literally translated `with men lying in bed like a woman.' The two roots taken together and pluralized, yield \textit{arsenokoites}. 

As Paul was a learned, Greek-speaking Jew, it is almost certain that he was familiar with text of the Septuagint. With no other uses of the word \textit{arsenokoites} previous to Paul's writing, it is most likely that Paul invented this compound word, as Shakespeare did with `eyeball,' `dishearten,' and `eventful' in English. Paul's use of this term is a direct link between the New Testament and the text of Leviticus. 

The two passages containing \textit{arsenokoites} are 
\begin{quote}{1 Corinthians 6:9-10, NRSV}
Do you not know that wrongdoers will not inherit the kingdom of God? Do not be deceived! Fornicators, idolaters, adulterers, male prostitutes, sodomites, thieves, the greedy, drunkards, revilers, robbers—none of these will inherit the kingdom of God. 
\end{quote}
\begin{quote}{1 Timothy 1:9-11, NRSV}
This means understanding that the law is laid down not for the innocent but for the lawless and disobedient, for the godless and sinful, for the unholy and profane, for those who kill their father or mother, for murderers, fornicators, sodomites, slave traders, liars, perjurers, and whatever else is contrary to the sound teaching that conforms to the glorious gospel of the blessed God, which he entrusted to me.
\end{quote}

In both contexts, \textit{arsenokoites} is used in a list of sins. In both cases it is listed adjacent to other sexually-related sins; adulterers and male prostitues in 1 Corintians and fornicators in 1 Timothy. It is most reasonable to translate this term, in this context, as referring to such men who have intercourse with other men. 

\section*{The account of Sodom}

There are many references to Sodom in the Bible and the deuterocanonical texts, quoted here in the NRSV translation. First are the `historical' references to the city before its destruction. The first reference, Genesis 14:1-14 discusses the war in which Sodom (and Gomorrah) participated participated. Sodom was on the losing side of the war; Lot, nephew of Abraham, was carried off as spoils of war when Sodom was sacked by its victorious enemies. Abraham lead his men to the rescue of Lot. No particular sinfulness is ascribed to Sodom at this time. 

The next reference is to the depravity of Sodom, and its eventual destruction. First, the Lord declares that Sodom is wicked and determines that He will investigate.
\begin{quote}{Genesis 18:20-21, NRSV}
Then the Lord said,``How great is the outcry against Sodom and Gomorrah and how very grave their sin! I must go down and see whether they have done altogether according to the outcry that has come to me; and if not, I will know''
\end{quote}
Next, Abraham pleads with the Lord to spare the city if righteous men can be found, bargaining with the Lord to spare the city if ten righteous men can be found in it.
\begin{quote}{Genesis 18:32, NRSV}
Then he said, ``Oh do not let the Lord be angry if I speak just once more. Suppose ten are found there.” He answered, “For the sake of ten I will not destroy it.''
\end{quote}
Finally, the angels that God has sent into the city are confronted by an angry crowd
\begin{quote}{Genesis 19:45, NRSV}
But before they lay down, the men of the city, the men of Sodom, both young and old, all the people to the last man, surrounded the house; and they called to Lot, ``Where are the men who came to you tonight? Bring them out to us, so that we may know them.''
\end{quote}
The `historical' account of the destruction of Sodom ascribes to them many sins, but when the Lord sends his messengers to Sodom to investigate, they are specifically confronted with a threat of non-consensual, homosexual sex. 

After the details of what happened to spur the city's destruction, many later biblical and extra biblical authors discussed the crimes of the city, offering many explanations for what the inhabitants did wrong.
\begin{quote}{Isaiah 3:9, NRSV}
The look on their faces bears witness against them;
    they proclaim their sin like Sodom,
    they do not hide it.
Woe to them!
    For they have brought evil on themselves.
\end{quote}
\begin{quote}{Jeremiah 23:14, NRSV}
But in the prophets of Jerusalem
    I have seen a more shocking thing:
they commit adultery and walk in lies;
    they strengthen the hands of evildoers,
    so that no one turns from wickedness;
all of them have become like Sodom to me,
    and its inhabitants like Gomorrah.
\end{quote}
\begin{quote}{Ezekiel 16:49-50, NRSV}
Now this was the sin of your sister Sodom: She and her daughters were arrogant, overfed and unconcerned; they did not help the poor and needy. They were haughty and did detestable things before me. Therefore I did away with them as you have seen.
\end{quote}
\begin{quote}{Jude 1:7, NRSV}
Likewise, Sodom and Gomorrah and the surrounding cities, which, in the same manner as they, indulged in sexual immorality and pursued unnatural lust, serve as an example by undergoing a punishment of eternal fire.
\end{quote}
The phrase `pursuing unnatural lust` deserves further attention. The original Greek text is {\textgreekfont ἀπελθοῦσαι ὀπίσω σαρκὸς ἑτέρας}; transliterated as \textit{apelthousai opiso sarkos eteras}. The base of word \textit{apelthousai} means `to go forth', the verb is used as an nominative, aorist, active participle. Aorist case means that the word refers to past events that still affect the present. A particple is a verb used as an adjective, while the nominative case refers to a word that is related to the subject of a sentence.  
The next word, \textit{opiso} is an adverb meaning `behind' or `after'. The word for the human body is \textit{sarkh}, where the `kh` is the Greek letter Xi; \textit{sarkos} is the genitive case of this word. The genitive case refers to a noun or adjective acts as a description of another word. In this case it is describing the verb `to go forth'. Finally \textit{eteras} is the genitive form of the adjective `other'. 

In the sentence of Jude 1:7, the simple subject-verb combination is `Sodom and Gomorrah serve'. Thus, \textit{apelthousai} describes Sodom as `having gone forth in the past.' The remaining words form a prepositional phrase describing the goings forth, they (the Sodomites) were going forth after other bodies. In the context of the story from Genesis, the other bodies are those of the messengers who were with Lot. Translations of this phrase include `pursued unnatural lust' in NRSV, `practiced unnatural vice' in NABRE, and `perversion' in NIV. The NRSV includes a note that a literal transltion would be `went after other flesh;' and NABRE has a note suggesting `went after alien flesh.' 

In addition to specific mentions of sin, there are a great many passages in the Bible comparing a threatened punishment with the fate of Sodom and Gomorrah. None of these passages associate Sodom with any particular sin, but instead warn that the fate of the sinful will be like (or worse than) Sodom and Gomorrah. These passages include Deuteronomy 29:23, Isaiah 1:9-10, Isaiah 13:19, Jeremiah 49:18, Jeremiah 50:40, Lamentations 4:6, Amos 4:11, Zephaniah 2:9, Matthew 10:15, Matthew 11:24, Luke 10:12, Luke 17:29, Romans 9:29 (which is a quotation by Paul of the aforementioned Isaiah 1:9), 2 Peter 2:6. The deuterocanonical books add to this Wisdom 10:6, Wisdom 19:17, and Sirach 16:8. Lastly, there is an allegorical reference to Sodom in Revelation 11:8. 

The ennumerated sins of Sodom found in the Bible thus include, pride in sinning, adultery, lying, assisting evildoers, arogance, gluttony, apathy, failure to assist the poor, sexual immorality, and `pursuing unnatural lust', in additional to the vignette of rape and male homosexual sex described in Genesis. Surely, the Lord is correct when he says of Sodom, `how very grave is their sin'!

By New Testament times, the sin that was most commonly associated with Sodom was that of homosexuality. The only New Testament discussion of Sodom's sin, as mentioned above, was in the book of Jude, which tells us that Sodom and Gomorrah were punshed due to `sexual immorality' and `going after other flesh.' 

Other Jewish literature of the early Christian era confirms that this view of Sodom's sins was widespread. Flavius Josephus, a Hellenic Jew in service of Rome, wrote \textit{The Antiquities of the Jews} around 93 AD, at about the same time as the final books of the New Testament were being composed. He writes:
\begin{quote}{Antiquities of the Jews, Book 1, Chapter 11:3-4}
Now when the Sodomites saw the young men to be of beautiful countenances, and this to an extraordinary degree, and that they took up their lodgings with Lot, they resolved themselves to enjoy these beautiful boys by force and violence...But God was much displeased at their impudent behaviour: so that he both smote those men with blindness, and condemned the Sodomites to universal destruction.
\end{quote}
In Josephus, the sin of Sodom is explicitly described, more directly than in the book of Genesis, to be lust towards young men and desire to rape them. Another account interpretation, by Philo, is even more explicit. Philo was a Hellenic Jewish philosopher who live contemporaneously with Jesus; perhaps from 20 BC to 50 AD. He wrote a series of works in Greek explaining and commenting on the lives of hte Patriarchs and Moses' Law. Philo provide background information on why Sodom was so sinful, 
\begin{quote}{On Abraham, 133-134}
The country of the Sodomites was a district of the land of Canaan, which the Syrians afterwards called Palestine, a country full of innumerable iniquities, and especially of gluttony and debauchery, and all the great and numerous pleasures of other kinds which have been built up by men as a fortress, on which account it had been already condemned by the Judge of the whole world. And the cause of its excessive and immoderate intemperance was the unlimited abundance of supplies of all kinds which its inhabitants enjoyed. For the land was one with a deep soil, and well watered, and as such produced abundant crops of every kind of fruit every year. And he was a wise man and spoke truly who said ``The greatest cause of all iniquity is found in overmuch prosperity." 
\end{quote}
The prosperity of the lands inclined its citizens to all sorts of gluttony, debauchery, and intemperance. But, according to Philo, it was one specific intemperance that lead to the city's destruction:
\begin{quote}{On Abraham, 137-138a}
But God, having taken pity on mankind, as being a Saviour and full of love for mankind, increased, as far as possible, the natural desire of men and women for a connexion together, for the sake of producing children, and detesting the unnatural and unlawful commerce of the people of Sodom, he extinguished it, and destroyed those who were inclined to these things, and that not by any ordinary chastisement, but he inflicted on them an astonishing novelty, and unheard of rarity of vengeance; for, on a sudden, he commanded the sky to become overclouded and to pour forth a mighty shower, not of rain but of fire
\end{quote}
Philo's view, as the most esteemed Jewish philosopher of his time, was that God intended for man's natural condition to be that men and women would desire a connection together; Sodom's great sin was in rejecting this desire, and thus Sodom's `unnatural commerce' lead to its novel and fiery destruction. 

By the Christian Era, the sin of Sodom, which had been variously described by prophets in the past, was increasingly viewed as unnatural lust, particularly between men. References to Sodom occur in the New Testament from five different writers; the authors of Matthew and Luke; Paul in the book of Romans, and in 2 Peter and Jude. In the context of contemporary Jewish writing, the sin evoked by reference to Sodom was the sin of homosexuality. 




%Of the many sins of Sodom, the (attempted) homosexual intercourse described in the story of its destruction was unique; no where else in the Bible is such a story found. The rape aspect of the incident, both threatened on the messengers and committed on Lot's daughters, is not unique in the Bible; there are stories of rapes of Tamar (2 Samuel 13:7-19) and a Levite concubine. Indeed, it can be argued that Lot's daughters, subjected to rape in the tale of Sodom, in turn rape their own father later (Genesis 19:30-38). The inhospitality aspect of the account of Sodom is also present elsewhere in the Bible. The people of Israel were denied passage through the lands of the Edomites (Numbers 


%The encounter at Sodom lead directly to the destruction of the city and that the destruction of the city was mentioned 14 (or 17) additional times in the scripture. This explains why homosexual intercourse is strongly associated with the story of Sodom, even though sexual perversion was only one of many sins of that city.

\section*{Epistle to the Romans}

\begin{quote}{Romans 1:26-27, SBL Greek New Testament}
\textgreekfont{Διὰ τοῦτο παρέδωκεν αὐτοὺς ὁ θεὸς εἰς πάθη ἀτιμίας· αἵ τε γὰρ θήλειαι αὐτῶν μετήλλαξαν\\τὴν φυσικὴν χρῆσιν εἰς τὴν παρὰ φύσιν, ὁμοίως τε καὶ οἱ ἄρσενες ἀφέντες τὴν φυσικὴν χρῆσιν\\τῆς θηλείας ἐξεκαύθησαν ἐν τῇ ὀρέξει αὐτῶν εἰς ἀλλήλους, ἄρσενες ἐν ἄρσεσιν τὴν ἀσχημοσύνην\\κατεργαζόμενοι καὶ τὴν ἀντιμισθίαν ἣν ἔδει τῆς πλάνης αὐτῶν ἐν ἑαυτοῖς ἀπολαμβάνοντες.}
\end{quote}
\begin{quote}{Transliteration, from above}
Dia touto paredoken autous o theos eis pathe atimias ai te gar theleiai auton metellakhan\\ ten phusiken chresin eis tes para phusin omoios te kai oi arsenes aphentes ten phusiken chresin\\tes theleias ekhekauthesan en te orekhei auton eis allelous arsenes en arsesin ten aschemosunen\\katergazomenoi kai ten autimistheian en edei tes planes auton en eautois apolambanontes
\end{quote}
\begin{quote}{Romans 1:26-27, NRSV}
For this reason God gave them up to degrading passions. Their women exchanged natural intercourse for unnatural, and in the same way also the men, giving up natural intercourse with women, were consumed with passion for one another. Men committed shameless acts with men and received in their own persons the due penalty for their error.
\end{quote}
There is relatively little ambiguity in this translation. %Here are some key phrases with literal translations, with words groupd in parenthesis corresponding with the original greek:
%\begin{quote}{}
%gar theleiai auton metellakhan ten phusiken chresin eis tes para phusin\\
%for femailes their exchanged the natural use for the against nature\\[0.1in]
%omoios te kai oi arsenes aphentes ten phusiken chresin tes theleias\\
%likewise and also the males leaving the natural use the female\\[0.1in]
%ekhekauthesan en te orekhei auton eis allelous arsenes en arsesin\\
%(were inflamed) by the lust for one another males with males
%\end{quote} 
From the context, it is clear that Paul believes the actions described are "degrading passions." After going on to list other sins, Paul informs us of the cost of such sinful behavior:
\begin{quote}{Romans 1:32, NRSV}
They know God’s decree, that those who practice such things deserve to die—yet they not only do them but even applaud others who practice them.
\end{quote}


\section*{Conclusions}

Sexual intercourse between men is explicitly condemned ``an abomination'' in the Old Testament law. The exacting details of Mosaic law do not apply to practicing Christians since Christ has come, but large parts of the Mosaic law overlap with the law of the Spirit, which Christians must follow, with the assistance with grace, if they love Christ. In two of Paul's letters, he includes a word formed from the Levitical proscription on male homosexual intercourse in a list of sins.


In the book of Romans, Paul directly refers to both lust and relations between members of the same sex as ``degrading'' and declares that those who practice such acts ``deserve to die.'' Not only this, but Paul also condemns those who would applaud those who practice these acts. In 1 Corinthians, Paul explicity states that those who commit thie sin of Leviticus, sexual intercourse between men, will not inherit the kingdom of God. In a third letter of Paul's, the sin of Leviticus is included in a list of sins to be avoided.

Finally, there are several references to the sinful city of Sodom in the New Testament. This sin of this city, at the time of the writing of the New Testament and especially in Hellenic Judaism, had come to be associated with sexual intercourse between men, as demonstrated by the writings of influential contemporary Jews such as Flavius Josephus and Philo. Indeed, the sin of Sodom is explicitly described as sexual immorality and unnatural lusts within the New Testament itself, by Jude. 

%These three strands of evidence converge to give us a clear picture of homosexual intercourse's place in the law of the Spirit. Homosexual intercourse is contrary to the law of the Spirit; and those who practice this activity, without repentence and application to Christ's grace, will not inherit the kingdom of Heaven. 

%\appendix
%\section*{Appendix: Translations of Leviticus 18:22 in the Sepuagint}\label{A}
%The base of the word transliterated as \textit{koimethese} is \textit{koimao}, which means literaly, to fall asleep, as when the Chief Priests order the tomb guards to report that disciples stole Jesus' body while they were asleep (Matt 28:13); and when the disciples slept at the Mount of Olives before Jesus' arrest (Luke 22:45). It is alternately used as a euphaism for death, as in the death of Stephen the martyr (Acts 7:60) and in a speech of Paul's when he describes that David `fell asleep' and `was laid beside his ancestors' (NRSV).

%In this usage in Levitacus, the verb has the second person, is passive and singular, has the indicative voice, and future tense; so `you will have fallen asleep'. The word transliterated \textit{koiten} is the accusative case, meaning it is the recipient of the action of the verb; the falling asleep happens in bed. The adjective \textit{gynaikeian} is accusative, meaning it is operating on the object of the verb, that is, the bed; it is the bed that is womanly. The final verb \textit{esti} is `to be'; in the third person, singuar, with indicative voice and present tense this is `it is'.

\end{document}